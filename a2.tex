\documentclass{article}

\usepackage[utf8]{inputenc}
\usepackage[style=apa, backend=biber]{biblatex}
\usepackage[a4paper, total={6in, 8in}]{geometry}
\usepackage{apalike}
\usepackage{setspace}
\usepackage{mathptmx}
\usepackage{bashful}
\bash
texcount -sum -1 a2.tex
\END

\renewcommand{\baselinestretch}{1.15}

\addbibresource{My Library.bib}

\doublespacing
\setlength{\parskip}{0.8em}
\setlength{\parindent}{0pt}
\title{Assignment 2}
\author{Jacob P }
\date{January 2024}

\begin{document}

\maketitle

\section{Introduction}
In the dynamic landscape of Human Resource Management (HRM), the strategic imperative to cultivate a diverse and inclusive workforce remains paramount. Amidst this backdrop, this report delves into the nuanced sphere of recruitment as it pertains to two distinct diversity groups: Indigenous Australians and older workers. Recognizing the former's rich cultural heritage and the latter's vast reservoirs of experience and wisdom, the report examines traditional and modern recruitment methodologies through a critical lense to unearth inherent biases and barriers that potentially impede the participation of these groups. By synthesizing the latest scholarly insights and empirical evidence, this analysis endeavors not only to spotlight systemic challenges but also to forge robust recommendations capable of sculpting equitable recruitment landscapes. Such refinements aim to not only enhance the fabric of organizational practice but align it with the evolving ethos of equality and opportunity in the Australian workforce.

\section{Literature Review and Analysis for Indigenous Australians}
The socioeconomic conditions of Indigenous Australians is a crucial matter within the Human Resource Management (HRM) sphere, research shows that this matter is sorely underlooked. \cite{pearsonExtendingBoundariesHuman2011} conveys this assessment, shedding light on the disproportionate challenges that impact Indigenous communities' engagement with HRM frameworks. Extending Pearson's conversation, \cite{podgerEnduringChallengesNew2017} presents a dire characterization of the economic challenges facing indigenous Australians, encapsulating this argument with the fact that Indigenous Australians are in a "Poverty Crisis" \parencite{podgerEnduringChallengesNew2017}. The research demonstrates that despite being land rich within Australia, they often struggle with financial instability.

This socioeconomic disadvantage manifests in multiple dimensions of Aboriginal life, such as the lack of access to the same standard healthcare, evidenced by a life expectancy that lags significantly behind the wider Australian population \parencite{leederAchievingEquityAustralian2003}. The implications of socioeconomic disadvantage extend to Aboriginal Australian's participation in the labour market. Further exacerbating this circumstance are the issues for those with disabilities, who encounter marked barriers in accessing necessary support systems \parencite{statisticsChapterDisability2011}. Within educational settings, Indigenous Australian students disproportionately endure experiences of bullying \parencite{coffinBullyingAboriginalContext2010}, which can severely impede their educational journey and, by extension, career prospects. 

\cite{rynesImportanceRecruitmentJob1991} positions recruitment as a critical driver for job choice, advocating that active and thoughtful recruitment strategies hold substantial value for both the prospective employee and the employer. Nonetheless, Indigenous Australians have often been sidelined by traditional recruitment protocols, which are predicated on mainstream credentials and experiences that may not align with or fully represent the competencies of Indigenous candidates \cite{pearsonExtendingBoundariesHuman2011}. The focus on formal qualifications, prior job experience, and references create implicit barriers, potentially overlooking unique skills and perspectives that Indigenous Australians can offer \parencite{pearsonExtendingBoundariesHuman2011}. This calls attention to the need for adaptive and culturally sensitive HRM practices, attuned to the diverse backgrounds and abiding strengths that Indigenous individuals bring to professional environments.

Reviewing the literature highlights the prevailing disconnect between current HRM strategies and the needs and talents of Indigenous Australians. It highlights the pressing requirement for recalibrating HR methodologies to be inclusive and responsive to the distinctive experiences of this group, carving out pathways that value their contributions and foster genuine opportunities for their participation in the workforce.

Despite the overwhelming issues that are present within the issue of Recruitment for Aboriginal Australian's some have come forth with potential solutions. \cite{raeDevelopingResearchPartnership2013} suggests that recruitment within the Aboriginal community is a multifaceted issue, which can be addressed with some key solutions. Firstly, as Aboriginal culture is a communal culture, it's imperitive to ensure there is an open consultation channel between the communities and the recruitment teams \parencite{raeDevelopingResearchPartnership2013}. Futhermore, it's important to respect the structure of Aboriginal society. Aboriginals have a deep connection and respect for local elders, and as such, consultation with those people should be a priority \parencite{raeDevelopingResearchPartnership2013}.

In summary, whilst the issue of Recruitment in the Aboriginal Australian community is deep and multifaceted, research shows that solutions are possible and should be implemented in order to ensure our workforce remains strong and a connected community. 

\section{Literature Review and Analysis for Older Workers}
\cite{lievensRecruitingHiringOlder2012} states that the workforce is constantly evolving. In recent times, labour markets have gotten much tighter, and with much smaller budgets \cite{lievensRecruitingHiringOlder2012}. Organisations are pivoting to use attractiveness as a means of pulling more labour whilst the market is so restricted \cite{lievensRecruitingHiringOlder2012}. This has caused alot of challenges for older workers, as companies shift to higher attractiveness, the older labourforce is having positions substitued by younger participants \parencite{drydakisInclusiveRecruitmentHiring2017}. \cite{drydakisInclusiveRecruitmentHiring2017} backs this claim up by stating that 67\% of older workers have at some point during the recent years experienced age-based descrimination. The aim of this section is to identify why this statistic is so high, and how this can be countered.

\cite{walkerCombatingAgeDiscrimination1999} describes that the European Union has identified age descrimination as a major problem, and as such have established the "Age Barriers Project". The Age Barriers Project was a European inititive to bolster the dwindiling older workforce in it's member states \parencite{walkerCombatingAgeBarriers1998}. As part of the project multiple EU Member states reviewed their aging workforce, outline the perspectives of their workforce by workshopping, derived incentives that can be used to combat the perspectives, and then provide practical implementation documentation of how they will implement the changes \parencite{walkerCombatingAgeBarriers1998}. \cite{walkerCombatingAgeDiscrimination1999} states that the project was a resounding success. It was idenitified that a large majority of the problems were work-place based \parencite{walkerCombatingAgeDiscrimination1999}. 

\cite{marchiondoDevelopmentValidationWorkplace2016} identifies the largest causes of work-place based problems involving older workers in their research. They identified that the modern workforce holds many stereotypes for older workers \parencite{marchiondoDevelopmentValidationWorkplace2016}. Some common ones are that they are less cometent, less able, and less proficient that nother works \parencite{marchiondoDevelopmentValidationWorkplace2016}. They go further to describe the stereotypes for older workers as being "Overwhelmingly Negative" \parencite{marchiondoDevelopmentValidationWorkplace2016}.

\section{Recommendations}
Practical Recommendations: Based on research, provide actionable recommendations for HR to implement at strategic or operational levels.

Practical Recommendations: Offer well-researched solutions for HR to address diversity issues for the second group.

\section{Conclusion}
Part 6: Conclusion (~100 words)

Recap the central points of your analysis and underscore the importance of your recommendations.
\section{Reference List}
\printbibliography
This document has \emph{\bashStdout} words.
\end{document}
