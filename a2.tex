\documentclass{article}

\usepackage[utf8]{inputenc}
\usepackage[style=apa, backend=biber]{biblatex}
\usepackage[a4paper, total={6in, 8in}]{geometry}
\usepackage{apalike}
\usepackage{setspace}
\usepackage{mathptmx}
\usepackage{bashful}
\bash
texcount -sum -1 a2.tex
\END

\renewcommand{\baselinestretch}{1.15}

\addbibresource{My Library.bib}

\doublespacing
\setlength{\parskip}{0.8em}
\setlength{\parindent}{0pt}
\title{Assignment 2}
\author{Jacob P }
\date{January 2024}

\begin{document}

\maketitle

\section{Introduction}
In the dynamic landscape of Human Resource Management (HRM), the strategic imperative to cultivate a diverse and inclusive workforce remains paramount. Amidst this backdrop, this report delves into the nuanced sphere of recruitment as it pertains to two distinct diversity groups: Indigenous Australians and older workers. Recognizing the former's rich cultural heritage and the latter's vast reservoirs of experience and wisdom, the report examines traditional and modern recruitment methodologies through a critical lense to unearth inherent biases and barriers that potentially impede the participation of these groups. By synthesizing the latest scholarly insights and empirical evidence, this analysis endeavors not only to spotlight systemic challenges but also to forge robust recommendations capable of sculpting equitable recruitment landscapes. Such refinements aim to not only enhance the fabric of organizational practice but align it with the evolving ethos of equality and opportunity in the Australian workforce.

\section{Literature Review and Analysis for Indigenous Australians}
According to \cite{pearsonExtendingBoundariesHuman2011}, the socioeconomic status of Indigenous Australians can only be described as disadvantaged. \cite{podgerEnduringChallengesNew2017} goes further to state that Indigenous Australians are currently in a state of "Poverty Crisis". They are a people that are Land Rich and Cash Poor \parencite{podgerEnduringChallengesNew2017}. Further, Indigenous Australians are more likely to suffer from lower life expectency \parencite{leederAchievingEquityAustralian2003}, less likely to access services if they have a disability \parencite{statisticsChapterDisability2011}, and more likely to be bullied in school \parencite{coffinBullyingAboriginalContext2010}. \cite{rynesImportanceRecruitmentJob1991} explains that recruitment active recruitment is highly important for job choice and will positively impact the business. Despite this, \cite{pearsonExtendingBoundariesHuman2011} goes on to describe as modern recruitment being traditional and not at all geared to assist in the recruiement of indigeneous australians.
\begin{itemize}
	\item Socioeconomic Status: Discuss the disadvantaged socioeconomic status of Indigenous Australians, especially in remote communities, and how it impacts employment equality 
	\item  Barriers to Employment: Outline barriers such as linguistic disadvantages in reading and writing English, health problems, poor housing, and lower consumer capacity that contribute to fewer employment prospects
\end{itemize}
Part 2: Literature Review and Analysis for Diversity Group 1 (~700-750 words) 
Group Overview: Identify issues unique to the Indigenous Australian group in the context of the chosen HRM area.
Literature Review: Discuss theories and empirical findings from at least four peer-reviewed sources.
Current Practices Analysis: Examine what practices are currently in place to resolve these issues.

\section{Literature Review and Analysis for Older Workers}
Group Overview: Discuss the second chosen diversity group's issues in the context of the HRM area.
Literature Review: Provide a synthesis of theories and empirical findings from at least four peer-reviewed sources.
Current Practices Analysis: Look at existing practices addressing the second group’s issues.

\section{Recommendations}
Practical Recommendations: Based on research, provide actionable recommendations for HR to implement at strategic or operational levels.

Practical Recommendations: Offer well-researched solutions for HR to address diversity issues for the second group.

\section{Conclusion}
Part 6: Conclusion (~100 words)

Recap the central points of your analysis and underscore the importance of your recommendations.
\section{Reference List}
Part 7: Reference List

Adhere to a report-style format with headings and subheadings.
Use Times New Roman size 12 font and double-spacing.
Avoid direct quotes and bullet points (except for recommendations).
Include page numbers at the bottom middle of each page.
Word count is 2000 words (±10\%), excluding reference list but including in-text citations.
State the word count after the reference list.

\printbibliography
This document has \emph{\bashStdout} words.

\end{document}
