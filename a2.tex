\documentclass{article}

\usepackage[utf8]{inputenc}
\usepackage[style=apa, backend=biber]{biblatex}
\usepackage[a4paper, total={6in, 8in}]{geometry}
\usepackage{apalike}
\usepackage{setspace}
\usepackage{mathptmx}
\usepackage{bashful}
\bash
texcount -sum -1 a2.tex
\END

\renewcommand{\baselinestretch}{1.15}

\addbibresource{My Library.bib}

\doublespacing
\setlength{\parskip}{0.8em}
\setlength{\parindent}{0pt}
\title{Assignment 2}
\author{Jacob P }
\date{January 2024}

\begin{document}

\maketitle

\section{Introduction}
In the dynamic landscape of Human Resource Management (HRM), the strategic imperative to cultivate a diverse and inclusive workforce remains paramount. Amidst this backdrop, this report delves into the nuanced sphere of recruitment as it pertains to two distinct diversity groups: Indigenous Australians and older workers. Recognizing the former's rich cultural heritage and the latter's vast reservoirs of experience and wisdom, the report examines traditional and modern recruitment methodologies through a critical lense to unearth inherent biases and barriers that potentially impede the participation of these groups. By synthesizing the latest scholarly insights and empirical evidence, this analysis endeavors not only to spotlight systemic challenges but also to forge robust recommendations capable of sculpting equitable recruitment landscapes. Such refinements aim to not only enhance the fabric of organizational practice but align it with the evolving ethos of equality and opportunity in the Australian workforce.

\clearpage
\section{Literature Review and Analysis for Indigenous Australians}
The socioeconomic conditions of Indigenous Australians is a crucial matter within the Human Resource Management (HRM) sphere, research shows that this matter is sorely underlooked. \textcite{pearsonExtendingBoundariesHuman2011} conveys this assessment, shedding light on the disproportionate challenges that impact Indigenous communities' engagement with HRM frameworks. Extending Pearson's conversation, \cite{podgerEnduringChallengesNew2017} presents a dire characterization of the economic challenges facing indigenous Australians, encapsulating this argument with the fact that Indigenous Australians are in a "Poverty Crisis" \parencite{podgerEnduringChallengesNew2017}. The research demonstrates that despite being land rich within Australia, they often struggle with financial instability.

This socioeconomic disadvantage manifests in multiple dimensions of Aboriginal life, such as the lack of access to the same standard healthcare, evidenced by a life expectancy that lags significantly behind the wider Australian population \parencite{leederAchievingEquityAustralian2003}. The implications of socioeconomic disadvantage extend to Aboriginal Australian's participation in the labour market. Further exacerbating this circumstance are the issues for those with disabilities, who encounter marked barriers in accessing necessary support systems \parencite{statisticsChapterDisability2011}. Within educational settings, Indigenous Australian students disproportionately endure experiences of bullying \parencite{coffinBullyingAboriginalContext2010}, which can severely impede their educational journey and, by extension, career prospects. 

\textcite{rynesImportanceRecruitmentJob1991} positions recruitment as a critical driver for job choice, advocating that active and thoughtful recruitment strategies hold substantial value for both the prospective employee and the employer. Nonetheless, Indigenous Australians have often been sidelined by traditional recruitment protocols, which are predicated on mainstream credentials and experiences that may not align with or fully represent the competencies of Indigenous candidates \parencite{pearsonExtendingBoundariesHuman2011}. The focus on formal qualifications, prior job experience, and references create implicit barriers, potentially overlooking unique skills and perspectives that Indigenous Australians can offer \parencite{pearsonExtendingBoundariesHuman2011}. This calls attention to the need for adaptive and culturally sensitive HRM practices, attuned to the diverse backgrounds and abiding strengths that Indigenous individuals bring to professional environments.

Reviewing the literature highlights the prevailing disconnect between current HRM strategies and the needs and talents of Indigenous Australians. It highlights the pressing requirement for recalibrating HR methodologies to be inclusive and responsive to the distinctive experiences of this group, carving out pathways that value their contributions and foster genuine opportunities for their participation in the workforce.

Despite the overwhelming issues that are present within the issue of recruitment for Aboriginal Australian's some have come forth with potential solutions. \textcite{raeDevelopingResearchPartnership2013} suggests that recruitment within the Aboriginal community is a multifaceted issue, which can be addressed with some key solutions. Firstly, as Aboriginal culture is a communal culture, it's imperitive to ensure there is an open consultation channel between the communities and the recruitment teams \parencite{raeDevelopingResearchPartnership2013}. Futhermore, it's important to respect the structure of Aboriginal society. Aboriginals have a deep connection and respect for local elders, and as such, consultation with those people should be a priority \parencite{raeDevelopingResearchPartnership2013}.

In summary, whilst the issue of recruitment in the Aboriginal Australian community is deep and multifaceted, research shows that solutions are possible and should be implemented in order to ensure our workforce remains strong and a connected community. 

\clearpage
\section{Literature Review and Analysis for Older Workers}
\textcite{lievensRecruitingHiringOlder2012} states that the workforce is constantly evolving. In recent times, labour markets have become much tighter, and with significantly smaller budgets \parencite{lievensRecruitingHiringOlder2012}. Organisations are pivoting to use attractiveness as a means of drawing more labour while the market is so restricted \parencite{lievensRecruitingHiringOlder2012}. This has caused a lot of challenges for older workers, as companies shift to prioritise attractiveness, and the older workforce is seeing positions substituted by younger participants \parencite{drydakisInclusiveRecruitmentHiring2017}. \cite{drydakisInclusiveRecruitmentHiring2017} underscores this claim by stating that 67\% of older workers have, at some point during recent years, experienced age-based discrimination. The aim of this section is to identify why this statistic is so high and how it can be countered.

\textcite{walkerCombatingAgeDiscrimination1999} describes that the European Union has identified age discrimination as a significant problem and, as such, has established the Age Barriers Project. The Age Barriers Project was a European initiative designed to bolster the dwindling older workforce in its member states \parencite{walkerCombatingAgeBarriers1998}. As part of the project, multiple EU Member States reviewed their ageing workforce \parencite{walkerCombatingAgeBarriers1998}. Outlined the perspectives of their labour force through workshops, derived incentives that can be used to combat these perspectives, and then provided practical implementation documentation on how they will implement the changes \parencite{walkerCombatingAgeBarriers1998}. \cite{walkerCombatingAgeDiscrimination1999} states that the project was a resounding success. It was identified that the majority of the problems were workplace-based \parencite{walkerCombatingAgeDiscrimination1999}.

\textcite{marchiondoDevelopmentValidationWorkplace2016} identifies the largest causes of workplace-based problems involving older workers in their research. It was found that the modern workforce holds many stereotypes about older workers \parencite{marchiondoDevelopmentValidationWorkplace2016}. Some common examples include being less competent, less capable, and less proficient than other workers \parencite{marchiondoDevelopmentValidationWorkplace2016}. It was further described that the stereotypes of older workers are "overwhelmingly negative" \parencite{marchiondoDevelopmentValidationWorkplace2016}. \cite{marchiondoDevelopmentValidationWorkplace2016} also suggests that management could potentially be at fault as well. The research identifies that management often projects stereotypes onto other employees, causing lower performance and work ethic in older workers due to discouragement \parencite{marchiondoDevelopmentValidationWorkplace2016}.

Despite the labor market's tendency to prioritize younger workers, it is essential to recognize and counteract the adverse impacts of this shift on older workers. Renowned studies have emphasized that an overwhelming majority of older employees face age-based discrimination, attributed to prevailing negative stereotypes. These stereotypes question their competence and work capacity, often leading to a decline in performance and motivation due to management's reinforcement. However, successful programs like the EU's Age Barriers Project demonstrate that systematic intervention and the cultivation of diverse, age-inclusive workplaces can lead to significant improvements. Therein lies the opportunity for organizations to re-evaluate and refine their practices to harness the full potential of the older workforce.

\clearpage
\section{Recommendations}

The recommendations provided are based on empirical research and are tailored to address the nuanced challenges that Indigenous Australians and the Older Workforce confront within the recruitment landscape. By synthesizing scholarly insights with strategic HRM practices, these recommendations aim to counter disparities and to harness the strengths of these diverse groups.

\subsection{Refined Recruitment Strategies for Indigenous Australians}

The research, spearheaded by authors such as \textcite{pearsonExtendingBoundariesHuman2011} and \cite{raeDevelopingResearchPartnership2013}, offers critical insights into the unique cultural and social constructs that shape the lived experiences of Indigenous Australians. These insights compel HRM professionals to cultivate adaptive strategies and culturally enlightened practices, as detailed below:

\begin{enumerate}
    \item \textbf{Cultural Competence and Recruitment Adaptability}: Indigenous Australians bring rich cultural knowledge and life experiences that conventional recruitment methods may overlook. A recruitment framework, as postulated by \textcite{pearsonExtendingBoundariesHuman2011}, should include cultural competence training for HR personnel, equipping them to adapt recruitment strategies that are reflective of the unique competencies and backgrounds of Indigenous applicants. Such training fosters cultural sensitivity, leading to informed and flexible HR practices.
    
    \item \textbf{Engagement With Indigenous Communities}: \textcite{raeDevelopingResearchPartnership2013} endorses community engagement as the crux of culturally relevant recruitment strategies. HR professionals should prioritize creating consultative partnerships with Indigenous communities. These consultations groups form initiatives that are not only culturally respectful but are also embraced by Indigenous prospective employees, ensuring Person-organization fit that aligns with communal values and the mission of the organization.

    \item \textbf{Recognition of Non-traditional Credentials}: \textcite{pearsonExtendingBoundariesHuman2011}'s work underscores the importance of redefining job criteria to consider diverse forms of knowledge and education that Indigenous Australians may present. By broadening the scope of credentials and competencies valued during recruitment, organizations can mitigate systemic barriers that have historically leveled the playing field, in order to promote diversity and equality.

    \item \textbf{Inclusion of Indigenous Elders in Recruitment}: Incorporating Indigenous elders into the recruitment process, as suggested by \textcite{raeDevelopingResearchPartnership2013}, can serve as a bridge between traditional cultural perspectives and contemporary organizational contexts. Elders can offer wisdom and insight that enrich HR practices and promote cultural approach to attracting and evaluating talent from Indigenous communities.

    \item \textbf{Tailored Indigenous Employment Initiatives}: Implementing specialized internship and mentorship programs, drawing from both \textcite{pearsonExtendingBoundariesHuman2011}'s and \cite{raeDevelopingResearchPartnership2013}'s recommendations, can create meaningful pathways for Indigenous peoples to access, grow, and lead within organizations. These programs signal an organization's commitment to the professional development and upward mobility of Indigenous employees, thus enhancing job attraction and retention.
\end{enumerate}

\subsection{Inclusive Recruitment Framework for the Older Workforce}

Research by \textcite{marchiondoDevelopmentValidationWorkplace2016} and findings from the 'Age Barriers Project' \parencite{walkerCombatingAgeDiscrimination1999} have illuminated the importance of dismantling age-related barriers in the context of HRM. The following recommendations are designed to leverage the  evidence of fostering a more responsive recruitment framework for older workers:

\begin{enumerate}
    \item \textbf{Dismantling Ageism in Recruitment}: Ageism can influence recruitment decisions, rendering older candidates at a disadvantage. By adopting strategies that \textcite{marchiondoDevelopmentValidationWorkplace2016} emphasizes, organizations should adopt robust anti-discrimination policies that explicitly counter age-based biases. These policies would highlight the importance of experience and the multitude of strengths that older workers contribute, thereby ensuring equitable treatment throughout the recruitment process.

    \item \textbf{Facilitation of Continuous Professional Development}: As workforce demographics evolve, so too should the opportunities for lifelong learning. Underpinning recruitment with development programs tailored to older workers affirms the principle that professional growth is perpetual. HRM strategies that prioritize ongoing skill development, as recommended by age workplace studies, not only aid in retaining relevance and proficiency amongst older workers but also foster a culture of continuous learning across the organization.

    \item \textbf{Promotion of Intergenerational Teams}: The virtues of age diversity within teams extends beyond just compliance with non-discrimination standards; it is about actively valuing the confluence of diverse age-related perspectives. Informed by the 'Age Barriers Project' \parencite{walkerCombatingAgeDiscrimination1999}, strategic team composition should celebrate and harness intergenerational collaboration, thereby enriching the organizational knowledge base and fostering a harmonious and innovative work environment.

    \item \textbf{Valuation of Experience in Recruitment Criteria}: Modifying recruitment criteria to recognize the intrinsic value of an older worker's experience can reposition the organization as an attractive employer for all ages. By integrating the perspective that seasoned professionals bring advanced levels of reliability, profound industry insights, and a mature work ethic, HRM can attract applicants who not only meet the required competencies but also elevate the collective expertise within the workplace.
\end{enumerate}

By interlacing the above synthesis of research findings with refined strategic HRM recruitment practices, organizations can actively contribute to a shift that not only elevates Indigenous Australians and the Older Workforce but also redefines the value proposition of diversity within the Australian labor market. The proposed recommendations aim not only to surmount existing recruitment challenges but also to lay the foundations for inclusive growth and sustained organizational success.

\clearpage
\section{Conclusion}
This report, guided by research and on current HRM practices, unveils the stark disparities confronting Indigenous Australians and older workers within recruitment. Leveraging research, it outlines actionable recommendations coherent with the goal of sculpting a recruitment paradigm that bridges the divide. Driven by the urgency to foster inclusive workplaces, the proposed reforms underscore the strategic realignment of HRM towards appreciating the inherent cultural wealth and accrued wisdom that these cohorts contribute. It is through the materialization of these recommendations that organizations will not only endorse but enact the ideals of diversity and equality, sculpting a workforce reflective of the broader society’s tapestry. Through such endeavors, we can aspire to foster a collaborative, diverse, and vibrant Australian labor market, wherein the capabilities and dignity of every individual are meritoriously celebrated and harnessed.

\clearpage
\printbibliography
\clearpage
This document has \emph{\bashStdout} words.
\end{document}
